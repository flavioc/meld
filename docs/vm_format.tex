\documentclass{article}
\oddsidemargin  0   in
\evensidemargin 0   in
\topmargin     -0.5 in
\textwidth      6.5 in
\textheight     9   in


\newenvironment{byte}%
{\begin{tabular}{|c|c|c|c|c|c|c|c|}\hline}%
{\hline\end{tabular}}%

\begin{document}

\newcommand{\inst}[3]{#1&\byte#2\\\endbyte&{\it #3}\\}
\newcommand{\op}[2]{\ensuremath{#1}&\byte#2\\\endbyte\\}
\newcommand{\val}[3]{#1&\byte#2\\\endbyte&{\it #3}\\}


A VM program consists of a single byte dictating the number of distinct tuples,
followed by one byte with the offset for each \emph{tuple descriptor}, followed by
the tuple descriptors, and, finally, the list of instructions that represent the
routines for processing each tuple.

A tuple descriptor consists of a 2 byte short indicating the offset into the
VM at which the tuple processing function begins, followed by 1 byte describing the tuple's properties and then
1 byte indicating the aggregate's type (what aggregate and which field, high and low nibble,
respectively), if any. Next there is a byte indicating the tuples order
of processing for stratification.
Next, there is one byte indicating the tuple's number of fields and a byte with the number of delta tuples.
These 7 bytes are followed by a sequence bytes, one for each argument, indicating the type of the tuple's
argument. Following this sequence, there may be
a sequence of 2 byte shorts for the delta tuples, where the first byte gives the id of the
delta tuple and the second byte indicates the delta field.

\begin{tabular}{lll}
INSTRUCTION & BYTE FORMAT & ARGS\\
\hline
\\
\inst{IF}    {0&1&1&0&0&0&0&0 \\\hline
0&0&0&r&r&r&r&r \\\hline
j&j&j&j&j&j&j&j \\\hline j&j&j&j&j&j&j&j} {reg, jump\_offset}
& if {\it reg} != 0 then process until ELSE and then jump.\\
& if {\it reg} = 0 then jump to ELSE \\
& (note: IFs may be nested) \\
\\
\inst{ELSE}  {0&0&0&0&0&0&1&0} {---}
& a marker for if blocks\\
\\
\inst{ITER}  {1&0&1&0&0&0&0&0\\\hline 0&i&i&i&i&i&i&i \\\hline
j&j&j&j&j&j&j&j \\\hline j&j&j&j&j&j&j&j} {id, jump\_offset, matchlist}
& iterates over all the tuples of type {\it id} that match\\
& according to the following {\it matchlist}.\\
& after all matching facts have been processed, use \\
& {\it jump\_offset} to jump to the next instruction\\
\\
\inst{NEXT}  {0&0&0&0&0&0&0&1} {---}
& return to iter and process next matching fact\\
\\
\inst{SEND}  {0&0&0&0&1&0&0&0\\\hline
0&0&0&r$_1$&r$_1$&r$_1$&r$_1$&r$_1$\\\hline
0&0&0&r$_2$&r$_2$&r$_2$&r$_2$&r$_2$\\\hline
0&0&v$_1$&v$_2$&v$_3$&v$_4$&v$_5$&v$_6$} {reg$_1$, reg$_2$, val}
& sends the tuple in {\it reg$_1$} along the path in {\it reg$_2$}\\
& if {\it reg$_1$ = reg$_2$} then the tuple is stored locally\\
& waits for {\it val} time before transmitting\\
\\
\inst{REMOVE} {1&0&0&r&r&r&r&r} {reg}
& delete tuple stored in reg from database\\
\\
\inst{OP}    {1&1&0&0&0&0&0&0\\\hline
0&0&v$_1$&v$_1$&v$_1$&v$_1$&v$_1$&v$_1$\\\hline
0&0&v$_2$&v$_2$&v$_2$&v$_2$&v$_2$&v$_2$\\\hline
0&0&v$_3$&v$_3$&v$_3$&v$_3$&v$_3$&v$_3$\\\hline
0&0&0&o&o&o&o&o} {val$_1$, val$_2$, val$_3$, op}
& sets {\it val$_3$} = {\it val$_1$ op val$_2$}\\
\\
\inst{NOT}    {0&0&0&0&0&1&1&1\\\hline
0&0&v$_1$&v$_1$&v$_1$&v$_1$&v$_1$&v$_1$\\\hline
0&0&v$_2$&v$_2$&v$_2$&v$_2$&v$_2$&v$_2$} {val$_1$, val$_2$}
& sets {\it val$_2$} = {\it not val$_1$}\\
\\
\inst{MOVE}  {0&0&1&1&0&0&0&0\\\hline
0&0&v$_1$&v$_1$&v$_1$&v$_1$&v$_1$&v$_1$\\\hline
0&0&v$_2$&v$_2$&v$_2$&v$_2$&v$_2$&v$_2$} {val$_1$, val$_2$}
& copies {\it val$_1$} to {\it val$_2$}\\
\\
\end{tabular}

\begin{tabular}{lll}
INSTRUCTION & BYTE FORMAT & ARGS\\
\hline
\\
\inst{MOVE-NIL}  {0&1&1&1&0&0&0&0\\\hline
0&0&v&v&v&v&v&v} {val}
& sets {\it val} to the nil list\\
\\
\inst{TEST-NIL}    {0&0&0&0&0&0&1&1 \\\hline
0&0&v$_1$&v$_1$&v$_1$&v$_1$&v$_1$&v$_1$\\\hline
0&0&v$_2$&v$_2$&v$_2$&v$_2$&v$_2$&v$_2$} {val$_1$, val$_2$}
& v$_2$ = 1 if v$_1$ is nil. \\
& v$_2$ = 0 if v$_1$ is not nil. \\
\\
\inst{ALLOC} {0&1&0&0&0&0&0&0\\\hline
0&i&i&i&i&i&i&i\\\hline
0&0&v&v&v&v&v&v} {id, val}
& allocates a tuple of type {\it id} and stores it in {\it val}\\
\\
\inst{RETURN}{0&0&0&0&0&0&0&0} {---}
& finished processing the tuple - return\\
\\
\inst{CALL}  {0&0&1&0&0&0&0&0\\\hline
0&i&i&i&i&i&i&i\\\hline
0&0&0&r&r&r&r&r} {id, reg, args}
& call external function number {\it id} with {\it args} and store\\
&the result in {\it reg}\\
\\
\inst{CONS}    {0&0&0&0&0&1&0&0\\\hline
0&0&v$_1$&v$_1$&v$_1$&v$_1$&v$_1$&v$_1$\\\hline
0&0&v$_2$&v$_2$&v$_2$&v$_2$&v$_2$&v$_2$\\\hline
0&0&v$_3$&v$_3$&v$_3$&v$_3$&v$_3$&v$_3$} {val$_1$, val$_2$, val$_3$}
& sets {\it val$_3$} = {\it val$_1$ :: val$_2$}\\
\\
\inst{HEAD}    {0&0&0&0&0&1&0&1\\\hline
0&0&v$_1$&v$_1$&v$_1$&v$_1$&v$_1$&v$_1$\\\hline
0&0&v$_2$&v$_2$&v$_2$&v$_2$&v$_2$&v$_2$} {val$_1$, val$_2$}
& sets {\it val$_2$} = {\it head val$_1$}\\
\\
\inst{TAIL}    {0&0&0&0&0&1&1&0\\\hline
0&0&v$_1$&v$_1$&v$_1$&v$_1$&v$_1$&v$_1$\\\hline
0&0&v$_2$&v$_2$&v$_2$&v$_2$&v$_2$&v$_2$} {val$_1$, val$_2$}
& sets {\it val$_2$} = {\it tail val$_1$}\\
\\
\end{tabular}

\begin{tabular}{lll}
OP & BYTE FORMAT\\
\hline
\\
\op{float\neq}{0&0&0&0&0}
\op{int\neq}{0&0&0&0&1}
\op{float=}{0&0&0&1&0}
\op{int=}{0&0&0&1&1}
\op{float<}{0&0&1&0&0}
\op{int<}{0&0&1&0&1}
\op{float\leq}{0&0&1&1&0}
\op{int\leq}{0&0&1&1&1}
\op{float>}{0&1&0&0&0}
\op{int>}{0&1&0&0&1}
\op{float\geq}{0&1&0&1&0}
\op{int\geq}{0&1&0&1&1}
\op{float\%}{0&1&1&0&0}
\op{int\%}{0&1&1&0&1}
\op{float+}{0&1&1&1&0}
\op{int+}{0&1&1&1&1}
\op{float-}{1&0&0&0&0}
\op{int-}{1&0&0&0&1}
\op{float*}{1&0&0&1&0}
\op{int*}{1&0&0&1&1}
\op{float\div}{1&0&1&0&0}
\op{int\div}{1&0&1&0&1}
\op{addr\neq}{1&0&1&1&0}
\op{addr=}{1&0&1&1&1}
\op{int\ list\ =}{1&1&0&0&0}
\end{tabular}
\vspace{0.3in}\\

\begin{tabular}{lll}
VALUE & BYTE FORMAT & ARGS\\
\hline
\\
\val{REG}   {1&r&r&r&r&r} {reg}
\\
\val{TUPLE} {0&1&1&1&1&1} {---}
& refers to the tuple currently being processed\\
\\
\val{HOST\_ID}   {0&0&0&0&1&1} {---}
& refers to the node currently being processed\\
\\
\val{NIL}   {0&0&0&1&0&0} {---}
& the empty list\\
\\
\val{INT}   {0&0&0&0&0&1} {int}
& the next 4 bytes after the current instruction\\
& are an immediate integer to which this refers\\
\\
\val{FLOAT} {0&0&0&0&0&0} {float}
& the next 4 bytes after the current instruction\\
& are an immediate float to which this refers\\
\\
\val{FIELD} {0&0&0&0&1&0} {}
& the next two bytes after the current instruction\\
& indicate a field of a register in the following format:\\
\val{}{X&X&X&X&f&f&f&f\\\hline X&X&X&r&r&r&r&r} {field, reg}
& with {\it reg} indicating a register with a tuple value\\
& and {\it field} indicating the tuple's field number. \\
\end{tabular}
\vspace{0.3in}\\

\begin{tabular}{lll}
ARGS & BYTE FORMAT\\
\hline
\\
\val{VALUE}   {X&X&v&v&v&v&v&v} {value}
\end{tabular}
\vspace{0.3in}\\

\begin{tabular}{lll}
MATCHLIST & BYTE FORMAT\\
\hline
\\
\val{MATCHLIST}   {f&f&f&f&f&f&f&f\\\hline m&m&v&v&v&v&v&v} {field, marker, value}
& requires that the tuple's field {\it field} match {\it value}\\
& mm=11 if the match list is empty and mm=01 for the last \\
& entry in the list.\\
\end{tabular}\\

\vspace{0.3in}

\begin{tabular}{lll}
AGGREGATE & BYTE FORMAT\\
\hline
\\
\op{none}{0&0&0&0}
\op{first}{0&0&0&1}
\op{int\ max}{0&0&1&0}
\op{int\ min}{0&0&1&1}
\op{int\ sum}{0&1&0&0}
\op{float\ max}{0&1&0&1}
\op{float\ min}{0&1&1&0}
\op{float\ sum}{0&1&1&1}
\op{int\ set\_union}{1&0&0&0}
\op{float\ set\_union}{1&0&0&1}
\op{int\ list\ sum}{1&0&1&0}
\op{float\ list\ sum}{1&0&1&1}
\end{tabular}

\vspace{0.3in}

\begin{tabular}{lll}
TYPE & BYTE FORMAT\\
\hline
\\
\op{int}{0&0&0&0}
\op{float}{0&0&0&1}
\op{addr}{0&0&1&0}
\op{int\ list}{0&0&1&1}
\op{float\ list}{0&1&0&0}
\op{addr\ list}{0&1&0&1}
\op{int\ set}{0&1&1&0}
\op{float\ set}{0&1&1&1}
\op{type}{1&0&0&0}
\end{tabular}
\vspace{0.3in}\\

\begin{tabular}{lll}
PROPERTY & BYTE POSITION\\
\hline
\\
\op{aggregate}{1}
\op{persistent}{2}
\op{linear}{3}
\op{delete}{4}
\op{schedule}{5}
\end{tabular}
\vspace{0.3in}\\

\noindent
NOTES:\\
All offsets and lengths are given in bytes.
\end{document}

